\documentclass[11pt,a4paper]{article}
\usepackage[spanish,es-nodecimaldot]{babel}	% Utilizar español
\usepackage[utf8]{inputenc}					% Caracteres UTF-8
\usepackage{graphicx}						% Imagenes

\PassOptionsToPackage{hyphens}{url}
\usepackage[hidelinks]{hyperref}			% Poner enlaces sin marcarlos en rojo

\usepackage{fancyhdr}						% Modificar encabezados y pies de pagina
\usepackage{float}							% Insertar figuras
\usepackage[textwidth=390pt]{geometry}		% Anchura de la pagina
\usepackage[nottoc]{tocbibind}				% Referencias (no incluir num pagina indice en Indice)
\usepackage{enumitem}						% Permitir enumerate con distintos simbolos
\usepackage[T1]{fontenc}					% Usar textsc en sections
\usepackage{amsmath}						% Símbolos matemáticos

% Comando para poner el nombre de la asignatura
\newcommand{\asignatura}{Aprendizaje Automático}
\newcommand{\autor}{Vladislav Nikolov Vasilev}

% Configuracion de encabezados y pies de pagina
\pagestyle{fancy}
\lhead{Vladislav Nikolov, José María Sánchez}
\rhead{\asignatura{}}
\lfoot{Grado en Ingeniería Informática}
\cfoot{}
\rfoot{\thepage}
\renewcommand{\headrulewidth}{0.4pt}		% Linea cabeza de pagina
\renewcommand{\footrulewidth}{0.4pt}		% Linea pie de pagina

\begin{document}
\pagenumbering{gobble}

% Pagina de titulo
\begin{titlepage}

\begin{minipage}{\textwidth}

\centering

\includegraphics[scale=0.5]{img/ugr.png}\\

\textsc{\Large \asignatura{}\\[0.2cm]}
\textsc{GRADO EN INGENIERÍA INFORMÁTICA}\\[1cm]

\noindent\rule[-1ex]{\textwidth}{1pt}\\[1.5ex]
\textsc{{\Huge PROYECTO FINAL\\[0.5ex]}}
\textsc{{\Large Subtítulo práctica\\}}
\noindent\rule[-1ex]{\textwidth}{2pt}\\[3.5ex]

\end{minipage}

\vspace{0.5cm}

\begin{minipage}{\textwidth}

\centering

\textbf{Autores}\\ {\autor{}}\\{José María Sánchez Guerrero}\\[2ex]
\textbf{Rama}\\ {Computación y Sistemas Inteligentes}\\[2ex]
\vspace{0.3cm}

\includegraphics[scale=0.3]{img/etsiit.jpeg}

\vspace{0.7cm}
\textsc{Escuela Técnica Superior de Ingenierías Informática y de Telecomunicación}\\
\vspace{1cm}
\textsc{Curso 2018-2019}
\end{minipage}
\end{titlepage}

\pagenumbering{arabic}
\tableofcontents
\thispagestyle{empty}				% No usar estilo en la pagina de indice


\newpage

\setlength{\parskip}{1em}

\section{\textsc{Descripción del problema}}

El conjunto de datos con el que vamos a trabajar es el \textit{Image Segmentation Data Set}, creado por Vision Group de la Univesidad
de Massachusetts, y contiene una serie de características de casos extraídos al azar de una base de datos con 7 tipos de imágenes al
aire libre. Éstas imágenes fueron segmentadas manualmente para crear una clasificación para cada pixel. Cada instancia es una región
de $3 \times 3$.

Originalmente, nuestro conjunto de datos estaba ya dividido en training y test. Sin embargo, disponemos solo de 210 datos de
entrenamiento y 2100 de test. Como la cantidad de datos que tenemos para entrenamiento es muy inferior a la de test, y no existe
un motivo justificado por el que las particiones se hayan hecho de esta forma, hemos decidido juntar los datos y crear nuestras
propias particiones de entrenamiento y test. Esta división se volverá a comentar en secciones posteriores. 

Por tanto, si analizamos los dos conjuntos de datos de forma conjunta, podemos ver que disponemos de 2310 muestras con 19 atributos
cada una, los valores de los cuáles son todos valores reales. Tal y como están originalmente los datos, la primera columna se
corresponde con la clase y las 19 columnas restantes son los datos de entrada.

Según la información proporcionada por la descripción del conjunto de datos, la cuál puede ser encontrada en el repositorio UCI
\cite{bib:uci-repo}, existen 7 clases distintas, y hay 30 muestras de cada clase en el conjunto de entrenamiento y 300 en el conjunto
de test. Con lo cuál, tenemos que en total hay 330 muestras de cada clase si miramos los dos conjuntos de datos de forma conjunta. Por
tanto, podemos afirmar que cada clase está idénticamente representada en los datos de los que disponemos, y que no existe una clase
que esté representada en mayor o menor medida que el resto de ellas. 

Con todo esto dicho, vamos a proceder a analizar cada una de las 19 características mencionadas anteriormente, para ver que representa
cada una de ellas:

\begin{enumerate}
	\item \textit{Region-centroid-col}: la columna del píxel central de la región.
	\item \textit{Region-centroid-row}: la fila del píxel central de la región. 
	\item \textit{Region-pixel-count}: el número de píxeles en una región. Su valor siempre es 9. 
	\item \textit{Short-line-density-5}: resultados de un algoritmo de extracción de rectas que cuenta cuántas líneas de longitud 5
	(con cualquier orientación) con bajo contraste, menor o igual a 5, cruzan la región. 
	\item \textit{Short-line-density-2}: igual que \textit{Short-line-density-5} pero cuenta con líneas de alto contraste,
	mayor que 5. 
	\item \label{it:vegde} \textit{Vegde-mean}: mide el contraste de píxeles horizontalmente adyacentes en la región. Hay 6 valores,
	pero se da dan la media y la desviación típica. Este atributo se utiliza como un detector de borde vertical. 
	\item \textit{Vegde-sd:} Desviación típica del contraste de píxeles horizontalmente adyacentes en la región (ver \ref{it:vegde}).
	\item \label{it:hedge} \textit{Hedge-mean}: mide el contraste de los píxeles verticalmente adyacentes.
	Utilizado para la detección de líneas horizontales. Este atributo es el valor medio.
	\item \textit{Hedge-sd}: Desviación típica del contraste de los píxeles verticalmente adyacentes (ver \ref{it:hedge}). 
	\item \textit{Intensity-mean}: el promedio sobre la región de (R + G + B) / 3.
	\item \textit{Rawred-mean}: Promedio sobre la región del valor R. 
	\item \textit{Rawblue-mean}: Promedio sobre la región del valor B. 
	\item \textit{Rawgreen-mean}: Promedio sobre la región del valor G. 
	\item \textit{Exred-mean}: Mide el exceso de rojo: (2R - (G + B)).
	\item \textit{Exblue-mean}: Mide el exceso de azul: (2B - (G + R)).
	\item \textit{Exgreen-mean}: Mide el exceso de verde: (2G - (R + B)).
	\item \label{it:value} \textit{Value-mean}: Transformación 3D no lineal de RGB.
	\item \textit{Saturatoin-mean}: (ver \ref{it:value}).
	\item \textit{Hue-mean}: (ver \ref{it:value}).
\end{enumerate}

Con esto dicho, vamos a analizar cuáles son los datos de salida. Como estamos en un problema de clasificación, cada entrada
produce una salida que es una etiqueta. La etiqueta puede hacer referencia a una de las 7 clases que tiene el problema, las cuáles
son: \textit{brickface}, \textit{sky}, \textit{foliage}, \textit{cement}, \textit{window}, \textit{path} y \textit{grass}.
Para facilitar el de representación de clase, vamos a transformar posteriormente cada valor de la etiqueta a un número entero. Esto
se verá en su correspondiente sección, cuando se realice el análisis y transformación de los datos.

Y con esto dicho, vamos a comenzar con un primer paso muy importante, el cuál es el análisis de los datos para intentar descubrir
qué nos pueden ofrecer éstos.

\newpage

\section{\textsc{Análisis de los datos}}

\newpage

% Pagina de bibliografia
\begin{thebibliography}{5}

\bibitem{bib:uci-repo}
UCI. \textit{Image segmentation}
\\\url{http://archive.ics.uci.edu/ml/datasets/image+segmentation}

\end{thebibliography}

\end{document}

