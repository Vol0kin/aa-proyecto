\documentclass[11pt,a4paper]{article}
\usepackage[spanish,es-nodecimaldot]{babel}	% Utilizar español
\usepackage[utf8]{inputenc}					% Caracteres UTF-8
\usepackage{graphicx}						% Imagenes

\PassOptionsToPackage{hyphens}{url}
\usepackage[hidelinks]{hyperref}			% Poner enlaces sin marcarlos en rojo

\usepackage{fancyhdr}						% Modificar encabezados y pies de pagina
\usepackage{float}							% Insertar figuras
\usepackage[textwidth=390pt]{geometry}		% Anchura de la pagina
\usepackage[nottoc]{tocbibind}				% Referencias (no incluir num pagina indice en Indice)
\usepackage{enumitem}						% Permitir enumerate con distintos simbolos
\usepackage[T1]{fontenc}					% Usar textsc en sections
\usepackage{amsmath}						% Símbolos matemáticos

% Comando para poner el nombre de la asignatura
\newcommand{\asignatura}{Aprendizaje Automático}
\newcommand{\autor}{Vladislav Nikolov Vasilev}

% Configuracion de encabezados y pies de pagina
\pagestyle{fancy}
\lhead{Vladislav Nikolov, José María Sánchez}
\rhead{\asignatura{}}
\lfoot{Grado en Ingeniería Informática}
\cfoot{}
\rfoot{\thepage}
\renewcommand{\headrulewidth}{0.4pt}		% Linea cabeza de pagina
\renewcommand{\footrulewidth}{0.4pt}		% Linea pie de pagina

\begin{document}
\pagenumbering{gobble}

% Pagina de titulo
\begin{titlepage}

\begin{minipage}{\textwidth}

\centering

\includegraphics[scale=0.5]{img/ugr.png}\\

\textsc{\Large \asignatura{}\\[0.2cm]}
\textsc{GRADO EN INGENIERÍA INFORMÁTICA}\\[1cm]

\noindent\rule[-1ex]{\textwidth}{1pt}\\[1.5ex]
\textsc{{\Huge PROYECTO FINAL\\[0.5ex]}}
\textsc{{\Large Subtítulo práctica\\}}
\noindent\rule[-1ex]{\textwidth}{2pt}\\[3.5ex]

\end{minipage}

\vspace{0.5cm}

\begin{minipage}{\textwidth}

\centering

\textbf{Autores}\\ {\autor{}}\\{José María Sánchez Guerrero}\\[2ex]
\textbf{Rama}\\ {Computación y Sistemas Inteligentes}\\[2ex]
\vspace{0.3cm}

\includegraphics[scale=0.3]{img/etsiit.jpeg}

\vspace{0.7cm}
\textsc{Escuela Técnica Superior de Ingenierías Informática y de Telecomunicación}\\
\vspace{1cm}
\textsc{Curso 2018-2019}
\end{minipage}
\end{titlepage}

\pagenumbering{arabic}
\tableofcontents
\thispagestyle{empty}				% No usar estilo en la pagina de indice


\newpage

\setlength{\parskip}{1em}

\section{\textsc{Descripción del problema}}

El conjunto de datos con el que vamos a trabajar es el \textit{Image Segmentation Data Set}, creado por Vision Group de la Univesidad de Massachusetts, y contiene una serie de características de casos extraídos al azar de una base de datos con 7 tipos de imágenes al aire libre. Éstas imágenes fueron segmentadas manualmente para crear una clasificación para cada pixel. Cada instancia es una región de 3x3.

Originalmente, nuestro conjunto de datos estaba ya dividido en training y test, sin embargo, disponemos de 210 datos de entrenamiento por 2100 para test. Como está muy descompensado, vamos a juntar los dos en un dólo conjunto de datos y posteriormente realizaremos la división. Por tanto, podemos decir que nuestro conjunto de datos estará compuesto por 2310 datos y 19 atributos, los valores de los cuáles son todos números reales. Los datos de la primera columna se corresponden con la información de salida, mientras que las 19  columnas restantes pertenecen a los datos de entrada.

Veamos ahora a qué pertenece cada una de las 19 características mencionadas anteriormente, ordenadas por su número de columna:

\begin{enumerate}
	\item Region-centroid-col: la columna del píxel central de la región.
	\item Region-centroid-row: la fila del píxel central de la región. 
	\item Region-pixel-count: el número de píxeles en una región = 9. 
	\item Short-line-density-5: los resultados de un algoritmo de extracto de línea que cuenta cuántas líneas de longitud 5 (cualquier orientación) con bajo contraste , menor o igual a 5, recorre la región. 
	\item Short-line-density-2: igual que densidad de línea corta-5 pero cuenta con líneas de alto contraste, mayor que 5. 
	\item Vegde-mean: mide el contraste de píxeles horizontales adyacentes en la región. Hay 6, se dan la media y la desviación estándar. Este atributo se utiliza como un detector de borde vertical. 
	\item Vegde-sd: (ver 6) 
	\item Hedge-mean: mide el contraste de los píxeles adyacentes verticalmente. Utilizado para la detección de líneas horizontales. 
	\item Hedge-sd: (ver 8). 
	\item Intensity-mean: el promedio sobre la región de (R + G + B) / 3 
	\item Rawred-mean: el promedio sobre la región del valor R. 
	\item Rawblue-mean: el promedio sobre la región del valor B. 
	\item Rawgreen-mean: el promedio sobre la región del valor G. 
	\item Exred-mean: mida el exceso de rojo: (2R - (G + B)) 
	\item Exblue-mean: mida el exceso de azul: (2B - (G + R)) 
	\item Exgreen-mean: mida la verde en exceso : (2G - (R + B)) 
	\item Value-mean: transformación 3D no lineal de RGB. (El algoritmo se puede encontrar en Foley y VanDam, Fundamentals of Interactive Computer Graphics) 
	\item Saturatoin-mean: (ver 17)
	\item Hue-mean: (ver 17)
\end{enumerate}

Los datos de salida son etiquetas que hacen referencia a cada uno de los tipos de imágenes que tenemos. Los 7 tipos de imágenes son: brickface, sky, foliage, cement, window, path, grass. Para trabajar más fácilmente con ellos, vamos a transformar cada una de estas etiquetas en valores enteros del 0 al 6 respectivamente en el orden dado anteriormente.

La distribución de clases está hecha para que haya 330 datos exactos de cada uno de los tipos de imágenes.


\newpage

% Pagina de bibliografia
\begin{thebibliography}{5}

\bibitem{nombre-referencia}
Texto referencia
\\\url{https://url.referencia.com}

\end{thebibliography}

\end{document}

